\documentclass[10pt,a4paper,french,titlepage]{article}
\author{par Gabriel Caux et Léo Peyronnet : Binome numéro 3}
\title{La réussite des alliances}
\date{Février-Avril 2022}
\usepackage[utf8]{inputenc}
\usepackage[T1]{fontenc}
\usepackage{babel}
\usepackage{amsfonts}
\usepackage{amsthm}
\usepackage{mathtools}
\usepackage{amssymb}
\usepackage{listings}
\usepackage[pdftex]{hyperref}
\usepackage{amssymb}
\usepackage{pst-node}


\begin{document}
\maketitle
\tableofcontents
\section{La réussite des alliances: structuration du jeu}
\subsection{Rappel des règles du jeu}
\subsection{Création des fonctions (partie guidée)}
\section{Les Extensions: ajouts de fonctionnalités.}
\subsection{Vérification de la pioche}
\subsection{Statistiques - (fonction res$\_$multi$\_$simulation et statistiques$\_$nb$\_$tas)}
\subsection{Probabilités - (fonction proba et affiche$\_$proba)}
\section{Le debug$\_$mode}
\subsection{Naissance de la nécessité de pouvoir tester les fonctions}
\subsection{Structure du programme}
\subsection{Limites et Erreurs}
\section{Interface utilisateur: proposer un produit fini.}
\section{Annexes}



Le liens de notre gitlab est :\url{https://gitlab.isima.fr/lepeyronne/la-reussite-des-alliances}
\end{document}
