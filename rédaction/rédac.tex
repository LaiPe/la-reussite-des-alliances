\documentclass[10pt,a4paper,french,titlepage]{article}
\author{par Gabriel Caux et Léo Peyronnet : Binome numéro 3}
\title{La réussite des alliances}
\date{Février-Avril 2022}
\usepackage[utf8]{inputenc}
\usepackage[T1]{fontenc}
\usepackage{babel}
\usepackage{amsfonts}
\usepackage{amsthm}
\usepackage{mathtools}
\usepackage{amssymb}
\usepackage{listings}
\usepackage[pdftex]{hyperref}
\usepackage{amssymb}
\usepackage{pst-node}


\begin{document}
\maketitle
\tableofcontents
\section{La réussite des alliances: structuration du jeu}
\subsection{Rappel des règles du jeu}
Nous allons commencer l'explication de notre projet en rappelant rapidement les règles de la réussite des alliances :\\
Pour jouer à ce jeux on commence par prendre un paquet de carte puis on le mélange pour créer une pioche, ensuite on pioche les cartes une par une en les posant de gauche a droite faces visibles une fois que trois cartes ont été posées nous pouvons commencer à jouer. On procédera ainsi, on suppose que la carte la plus a gauche est la numéro une la carte à sa droite la numéro deux celle a sa droite numéro trois et ainsi de suite si la carte numéro trois a une  similarité (même couleur ou même valeur) avec la carte numéro une il y a alors alliance de cartes, on peut alors faire un saut dans ce cas la la carte qui est située entre nos deux cartes passe sur le tas de celle qui l'a précède. Si aucun saut n'est possible il faudra continuer à piocher les cartes jusqu'à ce qu'un ou plusieurs saut soit possible le jeu s'arrête lorsqu'il n'est plus possible de piocher une carte et qu'il n'est pas possible de faire de saut non plus. On compte alors le nombre de tas restant et selon le nombre de tas requis pour gagner on sait si oui ou non on a réussi.


\subsection{Création des fonctions (partie guidée)}
\section{Les Extensions: ajouts de fonctionnalités.}
\subsection{Vérification de la pioche}
\subsection{Statistiques - (fonction res$\_$multi$\_$simulation et statistiques$\_$nb$\_$tas)}
\subsection{Probabilités - (fonction proba et affiche$\_$proba)}
\section{Le debug$\_$mode}
\subsection{Naissance de la nécessité de pouvoir tester les fonctions}
\subsection{Structure du programme}
\subsection{Limites et Erreurs}
\section{Interface utilisateur: proposer un produit fini.}
\section{Annexes}



Le liens de notre gitlab est :\url{https://gitlab.isima.fr/lepeyronne/la-reussite-des-alliances}
\end{document}
